%! Author = Luis
%! Date = 26.01.2024

\chapter{Fazit}\label{ch:fazit}
Die urspr{"u}nglichen Anforderungen wurden umgesetzt.
Im Laufe der Zeit hatte ich noch zus{"a}tzliche Ideen, die noch nicht alle Umgesetzt sind.
Was genau umgesetzt oder verworfen wurde ist den UserStories in Abschnitt~\ref{subsec:user-storys} zu entnehmen.
F{"u}r mich pers{"o}nlich war das Hauptziel neue Sachen zu lernen und etwas zu Experimentieren.
Das ist mir auch gelungen.
Ob mein Projekt gro{\ss}e Verwendung f{"u}r andere hat bleibt wie schon in der Einleitung erw{"a}hnt abzuwarten.
Momentan erm{"o}glicht die Anwendung eine Abgeschw{"a}chte form des DynDns.
Anstelle von eines Domainnamen wird der Server nach der IP-Adresse zu einem Device Namen gefragt.
DynDns ist nat{"u}rlich besser, kostet aber evtl. etwas.
M{"o}chte man nur etwas ausprobieren oder hat eine Anwendung ohne Domainnamen k{"o}nnte IP-Share aber durchaus sinnvoll sein.
Eine Praktische erweiterung f{"u}r Websites w{"a}re hier eine automatische Umleitung von z.B. \url{ipshare.de/redirect/MeineAnwendung} zu \url{oeffentlicheip:5000/}.
Eine weitere Idee, die ich h{"a}tte, ist ein ``reverse Proxy`'' mit dem ohne Port-Forwarding Kontakt zum Server aufgebaut werden k{"o}nnte.
