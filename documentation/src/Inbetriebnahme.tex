%! Author = Luis
%! Date = 26.01.2024

\chapter{Inbetriebnahme}\label{ch:inbetriebnahme}
Zur Inbetriebnahme auf einem lokalen Rechner sind die nachfolgenden Schritte notwendig.

\section{Dateien Herunterladen}\label{sec:dateien-herunterladen}
Die Dateien {"u}ber~\url{https://github.com/janluisho/ipshare} als ZIP herunterladen oder clonen.
Und in das Verzeichnis navigieren.

\section{Pakete Installieren}\label{sec:parkete-installieren}
Ich benutze Anaconda und w{"u}rde daher ein neues Environment anlegen.

\lstset{basicstyle=\fontsize{7}{8}\selectfont\ttfamily}
\vspace{3mm}
\begin{lstlisting}
    conda create --name myenv --file .\environment.yml
\end{lstlisting}
\vspace{3mm}

Ist aber abh{"a}ngig vom Paketverwaltungssystem.
Vorausgesetzt man hat Python bereits installiert k{"o}nnen die Pakete auch mit pip installiert werden.

\vspace{3mm}
\begin{lstlisting}
    pip install -r requirements.txt
\end{lstlisting}
\vspace{3mm}

\section{Datenbank anlegen}\label{sec:datenbank-anlegen}
Um die Datenbank anzulegen muss die $db.py$ Datei einmal ausgef{"u}hrt werden.
Dabei kann es zu einem $sqlalchemy.exc.InvalidRequestError$ kommen.
Als Workaround m{"u}ssen in der $app/\_\_init\_\_.py$ die Blueprints gezeigt in Listings~\ref{lst:datenbank} tempor{"a}r auskommentiert werden.
\lstinputlisting[firstline=44, firstnumber=44, lastline=54, caption=views.py, label={lst:datenbank}]{../../app/__init__.py}

\section{Ausf{"u}hren}\label{sec:ausfuhren}
Ist die Datenbank angelegt kann die Datei $run.py$ ausgef{"u}hrt werden, um die Flask Anwendung zu starten.
Als einstiegs Punkt kann auf der Webseite die Ip-Adresse mittig im Bild angeklickt werden, um die Adresse als Visitor zu teilen.
Danach kann z.B. ein Account {"u}ber die REGISTER-Schaltfl{"a}che rechts oben angelegt werden.
Nach dem Teilen einer Adresse als eingeloggter User lohnt sich ein Klick auf das Stiftsymbol neben der Adresse.
Es sollte dann ein QR-Code und der API-Token angezeigt werden.

% Bei neuen Bibliotheken
% conda env export > environment.yml; conda list -e > requirements.txt
