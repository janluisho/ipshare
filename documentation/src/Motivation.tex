%! Author = Luis
%! Date = 26.01.2024

\chapter{Motivation und Anforderungen}\label{ch:motivation-und-anforderungen}
Musstest du schon einmal die Ip-Adresse von einem anderen Ger{"a}t per Hand abtippen?
Wahrscheinlich nicht und wenn hat es bestimmt auch nicht weh getan.
Aber w{"a}hre es nicht viel besser stattdessen an beiden Ger{"a}ten den Browser zu {"o}ffen und auf eine Webseite zu gehen,
welche die Ip-Adresse von einem Ger{"a}t zum anderen {"u}bertr{"a}gt.
``Nein, es dauert genauso lang'' h{"o}re ich dich sagen.
Wahrscheinlich hast du recht.
Na ja, ich habe trotzdem einige Monate an einer Webseite entwickelt, die im Prinzip das macht.
Es handelt sich bei IpShare also um ein overengineertes Clipboard.

\section{Projektidee}\label{sec:projektidee}
Spa{\ss} beiseite, tats{"a}chlich habe ich allerdings einen Anwendungsfall, indem ich so etwas brauche.
An unserer Hochschule bekommt man im Eduroam-WLAN eine {"o}ffentliche Ip-Adresse.
F{"u}r Experimente ist das als Student nat{"u}rlich ein Fest.
Jedenfalls wollte ich mich von einem Raspberry Pi aus einem 5G Netz zu meinem Laptop verbinden.
Dabei handelt es sich nebenbei um das Maveric-Projekt.
Das funktioniert auch da ich ja eine {"o}ffentliche Ip bekomme.
Nur {"a}ndert sich diese Ip gelegentlich.
Leider kann ich keine Angaben dazu machen, wann oder wie oft das Vorkommt.
Es w{"a}re nicht schlecht, automatisiert die neue Ip herauszubekommen, da ich nur umst{"a}ndlich Zugriff auf den Raspberry Pi habe.
W{"a}hrend ich das hier schreibe, ist das noch nicht erprobt worden, sollte aber dank Api funktionieren.
Diese Problematik gab mir jedenfalls die Idee f{"u}r das Projekt.

\section{Ziel}\label{sec:ziel}
Ob mein Projekt gro{\ss}e Verwendung f{"u}r andere hat bleibt abzuwarten.
Im Wesentlichen ging es darum mich mit Webentwicklung und im speziellen Flask zu besch{"a}ftigen.
Etwas zu experimentieren und dabei zu lernen.
Dementsprechend sind manche Designentscheidungen auch getroffen, einfach um zu experimentieren.
Manche Dinge sind daher auch nicht ganz einheitlich implementiert.

\section{bestehende L{"o}sungen}\label{sec:bestehende-losungen}
Es gibt bereits Webseiten, welche einem die eigene {"o}ffentliche Ip Adresse verraten.
Diese w{"a}hren z.B. whatismyipaddress.com\cite{whatismyipaddress} oder whatismyip\cite{whatismyip}.
Sie erlauben es aber nicht, die Adresse von einem anderen Ger{"a}t zu bekommen.

\section{Anforderungen}\label{sec:anforderungen}
Auf die Details m{"o}chte ich im Kapitel zu Planung und Entwurf~\ref{ch:planung-und-entwurf} eingehen.
Im groben sollen Ip-Adressen von einem Ger{"a}t geteilt werden k{"o}nnen um sie dann von einem andern Ger{"a}t abrufen zu k{"o}nnen.
Als Rahmenbedingung f{"u}r das Projekt war noch gegeben das es sich um eine multiuserf{"a}hige Webanwendung mit Datenbank handeln soll.
Die Entwicklung sollte nach einem Agilen-Vorgehensmodell erfolgen.


